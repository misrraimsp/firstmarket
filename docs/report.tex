\documentclass[a4paper]{report}

\author{Misrraim~Su\'arez~P\'erez}
\title{FirstMarket}

\usepackage{comment}
\begin{document}

    \maketitle
    \tableofcontents

    \section{An\'alisis}\label{sec:an'alisis}

        \subsection{Requisitos}\label{subsec:requisitos}

            La aplicaci\'on debe admitir los roles y capacidades siguientes:
            \begin{enumerate}
                \item Usuario an\'onimo (UA).
                \begin{itemize}
                    \item Por defecto, al acceder al sitio web se hace como UA, sin ninguna validacion ni credencial. Basta con acceder a la pagina (url) de inicio de la aplicacion.
                    \item Un UA debe poder realizar b\'usquedas de libros, es decir, debe tener pleno acceso a la exploracion del cat\'alogo.
                    \item La plena exploraci\'on del cat\'alogo debe permitir realizar b\'usque\-das filtradas seg\'un 0, 1 o m\'as criterios, tales como: categor\'\i{}a, t\'\i{}tulo o autor.
                    \item Un UA debe poder visualizar informacion detallada de un libro, por ejemplo de entre los obtenidos tras una busqueda.
                    \item Un UA debe poder consultar las promociones disponibles.
                    \item Un UA debe poder registrarse en el sistema completando un formulario (nombre, contrase\~na, direccion de correo electronico, etc.).
                    \item finalizado el proceso de registro, el nuevo UR debe recibir confirmacion por e-mail
                    \item en su caso, un UA debe poder hacer login en el sistema.
                    \item en su caso, un UA debe poder realizar el procedimiento de recuperacion de contrase\~na.
                \end{itemize}
                \item Usuario registrado (UR). Este perfil representa a un usuario que ha pasado de an\'onimo a registrado. Un UR posee todas las capacidades del UA, m\'as otras espec\'\i{}ficas suyas, a saber:
                \begin{itemize}
                    \item Poder editar la informacion de su perfil de usuario.
                    \item Realizar pedidos y efectuar los correspondientes pagos a traves de una pasarela segura.
                    \item Disponer de un carrito virtual para la gestion de la compra.
                    \item En el carrito se debe poder introducir, modificar la cantidad o eliminar libros (esto ultimo de uno en uno o todos a la vez).
                    \item en cualquier momento del proceso de realizar un pedido, el UR debe poder cancelarlo
                    \item Tras una compra, el UR debe recibir confirmacion en su correo electronico
                    \item un UR debe poder consultar el estado de sus pedidos
                    \item Puntuar (de alguna manera, p.e. estrellas del 1 al 5) un determinado libro que haya adquirido. Debe poder hacerlo en cualquier momento tras la compra.
                    \item Consultar un historico de sus transacciones, detallando los libros comprados, la fecha de la compra y el precio de cada uno.
                    \item Darse de baja como UR.
                    \item Cerrar sesion.
                    \item Un UR debe poder ponerse en contacto con el admin a traves de un formulario de contacto, recibiendo confirmacion por e-mail tras el envio del mismo.
                \end{itemize}
                \item Usuario Administrador
                \begin{itemize}
                    \item Ver y editar (a\~nadir, modificar, eliminar) la jerarquia de categorias (CRUD categorias).
                    \item Ver y editar (a\~nadir, modificar, eliminar) la informacion relativa a los libros (titulo, autor/es, editorial, precio, disponibilidad, \ldots) (CRUD libros).
                    \item Crear, modificar o eliminar promociones de libros (CRUD promos).
                    \item Tener acceso a la informacion de los UR, salvo sus contrase\~nas.
                    \item Dar de baja a un UR.
                    \item Visualizar la informacion de los pedidos, tanto los que esten en curso como los finalizados.
                    \item Poder alterar el estado de un pedido.
                    \item Poder generar informes (p.e. ventas durante un determinado periodo con su importe y la facturacion total).
                \end{itemize}
            \end{enumerate}

            Adem\'as de lo anterior, la aplicaci\'on debe:
            \begin{enumerate}
                \item[a)] garantizar la persistencia de los datos referentes a UR, pedidos, pagos, productos y sus categor\'\i{}as
                \item[b)] mostrar un mensaje de error cuando un usuario introduzca incorrectamente sus credenciales de autenticaci\'on
            \end{enumerate}

        \subsection{Casos de Uso}\label{subsec:casos-de-uso}

            \subsubsection{CU01\_RecuperarContrase\~na}
                Un usuario previamente registrado en el sistema intenta acceder al mismo, pero no recuerda su contrasena. El sistema intentar\'a crear una nueva contrasena y enviarsela al e-mail del usuario.
                \begin{itemize}
                    \item[+] Actores implicados: Usuario An\'onimo.
                    \item[+] Flujo principal:
                    \begin{enumerate}
                        \item El usuario hace click en la opci\'on \emph{?`olvid\'o su contrase\~na?}.
                        \item El sistema presenta un formulario donde introducir la direccion de e-mail.
                        \item El usuario introduce y envia su direccion de e-mail.
                        \item El sistema comprueba la direccion de e-mail.
                        \item Comprobacion correcta. El sistema envia un e-mail con un link de confirmacion a la direccion proporcionada, e informa de ello al usuario por pantalla.
                        \item Link no caducado. El usuario accede a su e-mail y hace click en el link, confirmando que efectivamente la direccion proporcionada es la suya.
                        \item El sistema genera una nueva contrasena y se la asigna al usuario.
                        \item El sistema envia la nueva contrasena a la direccion de e-mail del usuario, y le avisa de ello por pantalla.
                    \end{enumerate}
                    \item[+] Flujo alternativo: \emph{usuario\_no\_registrado}. La direccion proporcionada no se encuentra registrada en el sistema.
                    \begin{itemize}
                        \item[5.b.] El sistema no envia correo alguno pero, por razones de seguridad, desde el punto de vista del usuario este flujo alternativo es indistinguible del proncipal.
                    \end{itemize}
                    \item[+] Flujo excepcional: \emph{time\_out}. El usuario no confirma su direccion de e-mail dentro de un plazo determinado.
                    \begin{itemize}
                        \item[6.b.] El sistema detecta el timeout e invalida el link de confirmacion enviado. Si el usuario hace click en el link caducado se le informara de dicha condicion.
                    \end{itemize}
                \end{itemize}

            \subsubsection{CU02\_Alta}
                Proceso por el cual se crea un nuevo usuario registrado.
                \begin{itemize}
                    \item[+] Actores implicados: Usuario An\'onimo.
                    \item[+] Flujo principal:
                    \begin{enumerate}
                        \item[1.] El usuario hace click en el link de nuevo registro, disponible en la p\'agina de \emph{login}.
                        \item[2.] El sistema presenta un formulario donde introducir la direcci\'on de e-mail y la contrase\~na.
                        \item[3.] El usuario introduce y env\'\i{}a la informaci\'on pedida.
                        \item[4.] El sistema comprueba la informaci\'on proporcionada.
                        \item[5.] El sistema crea una nueva cuenta de usuario, pero la mantiene inactiva a la espera de confirmar la direcci\'on de email.
                        \item[6.] El sistema env\'\i{}a un email con un link de confirmaci\'on a la direcci\'on proporcionada, e informa al usuario por pantalla.
                        \item[7.] El usuario accede a su e-mail y hace click en el link enviado, confirmando que la direcci\'on de email es suya.
                        \item[8.] El sistema activa la cuenta de usuario.
                        \item[9.] El sistema env\'\i{}a al usuario un email de bienvenida.
                        \item[10.] El sistema redirige al usuario a la p\'agina de \emph{login}.
                    \end{enumerate}
                    \item[+] Flujo alternativo: \emph{usuario\_ya\_registrado}. La direcci\'on proporcionada se encuentra registrada en el sistema.
                    \begin{itemize}
                        \item[5.b.] El sistema no crea una nueva cuenta de usuario.
                        Por razones de seguridad, la manera en que el sistema informa al usuario tras esta situacion es indistinguible del flujo principal,
                        de forma que no se pueda deducir que ese email ya tiene cuenta asociada en el sistema.
                    \end{itemize}
                    \item[+] Flujo alternativo: \emph{link\_ya\_enviado}. El sistema est\'a a la espera de la confirmacion de un link valido en esta direccion de email.
                    \begin{itemize}
                        \item[5.b.] El sistema no crea una nueva cuenta de usuario. Se informa al usuario acerca de una condicion de error \-{}generica, por seguridad\- en relacion
                        con la direccion de email proporcionada, pidiendose que compruebe su direcci\'on de email.
                    \end{itemize}
                    \item[+] Flujo excepcional: \emph{time\_out}. El usuario no confirma su direccion de e-mail dentro de un plazo determinado.
                    \begin{itemize}
                        \item[7.b.] El sistema detecta el timeout e invalida el link de confirmacion enviado. Si el usuario hace click en el link caducado se le informara de dicha condicion.
                    \end{itemize}
                \end{itemize}

            \subsubsection{CU03\_Login}
                Proceso por el cual un usuario se autentica en el sistema.
                \begin{itemize}
                    \item[+] Actores implicados: Usuario An\'onimo.
                    \item[+] Flujo principal:
                    \begin{enumerate}
                        \item[1.] El usuario hace click en el link de \emph{login}, o intenta realizar alguna operaci\'on
                        que requiera autenticaci\'on (por ejemplo, a\~nadir un libro al carrito).
                        \item[2.] El sistema presenta el formulario de acceso (direcci\'on de email y la contrase\~na).
                        \item[3.] El usuario introduce y env\'\i{}a la informaci\'on pedida.
                        \item[4.] El sistema comprueba las credenciales.
                        \item[5.] El sistema redirige al usuario a la p\'agina principal.
                    \end{enumerate}
                    \item[+] Flujo alternativo: \emph{fallo\_autenticaci\'on}. El email no se encuentra registrado en el sistema, o la contrase\~na proporcionada no es correcta.
                    \begin{itemize}
                        \item[5.b.] El sistema informa al usuario acerca de un fallo gen\'erico de autenticaci\'on, de forma que,
                        por razones de seguridad, no se pueda deducir si el email proporcionado se encuentra registrado en el sistema.
                    \end{itemize}
                    \item[+] Flujo alternativo: \emph{bloqueo\_cuenta}. El usuario an\'onimo realiza, dentro de un marco de tiempo (configurable), un n\'umero de intentos (configurable) de login con contrase\~na err\'onea.
                    \begin{itemize}
                        \item[5.b.] El sistema bloquea la cuenta asociada al email para prevenir ataques por fuerza bruta.
                        \item[6.] El sistema informa al usuario por pantalla y mediante el env\'\i{}o de un email.
                        \item[7.] Pasado el tiempo de seguridad (configurable), el sistema desbloquea la cuenta del usuario.
                    \end{itemize}
                    \item[+] Post-Condiciones: El usuario pasa a tener rol de usuario registrado en el sistema.
                \end{itemize}

    

\end{document}
