\documentclass[a4paper,11pt]{report}

\author{Misrraim~Su\'arez~P\'erez}
\title{FirstMarket}

\begin{document}

    \maketitle
    \tableofcontents

    \section{An\'alisis}
    \subsection{Requisitos}
    La aplicaci\'on debe admitir los roles y capacidades siguientes:
    \begin{enumerate}
        \item Usuario an\'onimo (UA).
        \begin{itemize}
            \item Por defecto, al acceder al sitio web se hace como UA, sin ninguna validacion ni credencial. Basta con acceder a la pagina (url) de inicio de la aplicacion.
            \item Un UA debe poder realizar b\'usquedas de libros, es decir, debe tener pleno acceso a la exploracion del cat\'alogo.
            \item La plena exploracion del catalogo debe permitir realizar busquedas filtradas segun 0, 1 o mas criterios, tales como: categoria, titulo o autor.
            \item Un UA debe poder visualizar informacion detallada de un libro, por ejemplo de entre los obtenidos tras una busqueda.
            \item Un UA debe poder consultar las promociones disponibles.
            \item Un UA debe poder registrarse en el sistema completando un formulario (nombre, contrase\~na, direccion de correo electronico, etc.).
            \item finalizado el proceso de registro, el nuevo UR debe recibir confirmacion por e-mail
            \item en su caso, un UA debe poder hacer login en el sistema.
            \item en su caso, un UA debe poder realizar el procedimiento de recuperacion de contrase\~na.
        \end{itemize}
        \item Usuario registrado (UR). Este perfil representa a un usuario que ha pasado de an\'onimo a registrado. Un UR posee todas las capacidades del UA, m\'as otras espec\'\i{}ficas suyas, a saber:
        \begin{itemize}
            \item Poder editar la informacion de su perfil de usuario.
            \item Realizar pedidos y efectuar los correspondientes pagos a traves de una pasarela segura.
            \item  Disponer de un carrito virtual para la gestion de la compra.
            \item En el carrito se debe poder introducir, modificar la cantidad o eliminar libros (esto ultimo de uno en uno o todos a la vez).
            \item en cualquier momento del proceso de realizar un pedido, el UR debe poder cancelarlo
            \item Tras una compra, el UR debe recibir confirmacion en su correo electronico
            \item un UR debe poder consultar el estado de sus pedidos
            \item Puntuar (de alguna manera, p.e. estrellas del 1 al 5) un determinado libro que haya adquirido. Debe poder hacerlo en cualquier momento tras la compra.
            \item Consultar un historico de sus transacciones, detallando los libros comprados, la fecha de la compra y el precio de cada uno.
            \item Darse de baja como UR.
            \item Cerrar sesion.
            \item Un UR debe poder ponerse en contacto con el admin a traves de un formulario de contacto, recibiendo confirmacion por e-mail tras el envio del mismo.
        \end{itemize}
        \item Usuario Administrador
        \begin{itemize}
            \item Ver y editar (a\~nadir, modificar, eliminar) la jerarquia de categorias (CRUD categorias).
            \item Ver y editar (a\~nadir, modificar, eliminar) la informacion relativa a los libros (titulo, autor/es, editorial, precio, disponibilidad, \ldots) (CRUD libros).
            \item Crear, modificar o eliminar promociones de libros (CRUD promos).
            \item Tener acceso a la informacion de los UR, salvo sus contrase\~nas.
            \item Dar de baja a un UR.
            \item Visualizar la informacion de los pedidos, tanto los que esten en curso como los finalizados.
            \item Poder alterar el estado de un pedido.
            \item Poder generar informes (p.e. ventas durante un determinado periodo con su importe y la facturacion total).
        \end{itemize}
    \end{enumerate}

    Adem\'as de lo anterior, la aplicaci\'on debe:
    \begin{enumerate}
        \item[a)] garantizar la persistencia de los datos referentes a UR, pedidos, pagos, productos y sus categor\'\i{}as
        \item[b)] mostrar un mensaje de error cuando un usuario introduzca incorrectamente sus credenciales de autenticaci\'on
    \end{enumerate}

    \subsection{Casos de Uso}

    %Caso de uso: [The use case name and functional area/package it resides in.]

    %Descripci\'on: [A brief description of this Use Case Specification.]

    %Actores implicados:

    %Pre-Condiciones: [A constraint that must be true when a use case is invoked]

    %Flujo principal: [The part of a use case that describes its most common implementation.  The basic flow is written assuming that no errors or alternatives exist.  Provide a short description of what the base flow should accomplish.]

    %1. < workflow step>
    %2. The use case ends.

    %Flujo alternativo: <name of alternate workflow> [A brief description of the alternate workflow.]

    %1. < workflow step>
    %2. The use case ends.

    %Flujo excepcional: [The part of a use case that describes its error conditions.] <name of exception workflow> [A brief description of the exception workflow.]

    %1. < workflow step>
    %2. The use case resumes at step 2 of the base flow.

    %Requisitos especiales: [The special requirements of the use case are included here.  These requirements are not covered by the workflow as it has been described in the sections above.] <Name of Special Requirement> [A brief description of the special requirement.]

    %Post-Condiciones: [A constraint that must be true when a use case has ended]

    %Puntos de extensi\'on:
    %<<Includes>>
    %[In Unified Modeling Language (UML), a relationship from a base use case to an included use case specifying how the behavior defined for the included use case can be inserted into the behavior defined for the base use case.]
    %<<Extends>>
    %[In UML, a relationship from an extending use case to a base use case specifying how the behavior defined for the extending use case can be optionally inserted into the behavior defined for the base use case.]


    \subsubsection{CU01\_RecuperarContrase\~na}
    Un usuario previamente registrado en el sistema intenta acceder al mismo, pero no recuerda su contrasena. El sistema intentar\'a crear una nueva contrasena y enviarsela al e-mail del usuario.
    \begin{itemize}
        \item[+] Actores implicados: Usuario An\'onimo.
        \item[+] Flujo principal:
        \begin{enumerate}
            \item El usuario hace click en la opci\'on \emph{?`olvid\'o su contrase\~na?}.
            \item El sistema presenta un formulario donde introducir la direccion de e-mail.
            \item El usuario introduce y envia su direccion de e-mail.
            \item El sistema comprueba la direccion de e-mail.
            \item Comprobacion correcta. El sistema envia un e-mail con un link de confirmacion a la direccion proporcionada, e informa de ello al usuario por pantalla.
            \item Link no caducado. El usuario accede a su e-mail y hace click en el link, confirmando que efectivamente la direccion proporcionada es la suya.
            \item El sistema genera una nueva contrasena y se la asigna al usuario.
            \item El sistema envia la nueva contrasena a la direccion de e-mail del usuario, y le avisa de ello por pantalla.
        \end{enumerate}
        \item [+] Flujo alternativo: \emph{usuario\_no\_registrado}. La direccion proporcionada no se encuentra registrada en el sistema.
        \begin{itemize}
            \item [5.b.] El sistema no envia correo alguno pero, por razones de seguridad, desde el punto de vista del usuario este flujo alternativo es indistinguible del proncipal.
        \end{itemize}
        \item[+] Flujo excepcional: \emph{time\_out}. El usuario no confirma su direccion de e-mail dentro de un plazo determinado.
        \begin{itemize}
            \item [6.b.] El sistema detecta el timeout e invalida el link de confirmacion enviado. Si el usuario hace click en el link caducado se le informara de dicha condicion.
        \end{itemize}
    \end{itemize}


\end{document}
